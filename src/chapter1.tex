% !TEX TS-program = pdflatex
% !TEX root = ../QuantumDynamics.tex

\chapter{定态哈密顿量体系的时间演化}
\label{cpt:1}
\section{定态薛定谔方程的含时通解}
\begin{framed}
本节参考Schiff, QM 3rd ed., 2.6-8节(pp 19--37)及 Messiah QM, 2.15节(pp 68--71)。
\end{framed}

描述非相对论体系下微观粒子运动的薛定谔方程通式为
\begin{equation}\label{eqn:tdsefull}
i\hbar \dfrac{\pd \psi(\br,t)}{\pd t}=\bH(\br,t) \psi(\br,t)
\end{equation}
其中$ \bH $为哈密顿量,可能与时间有关,因此此方程亦称为\textbf{含时薛定谔方程};$ \Psi(\br,t) $为波函数,其模的平方称为概率密度
\begin{equation}\label{key}
P(\br,t)=\psi^*(\br,t)\psi(\br,t)=|\psi(\br,t)|^2
\end{equation}
通常要求波函数归一化,即
\begin{equation}\label{key}
\int |\psi(\br,t)|^2d\tau=1
\end{equation}
在本书中,我们主要研究单粒子行为,因此波函数均为单粒子波函数。

哈密顿量$ \bH $在坐标表象下通常可以写为两项之和
\begin{equation}\label{key}
\bH=-\dfrac{\hbar^2}{2m}\nabla^2+V(\br,t)
\end{equation}
通常第一项动能部分与时间无关,而第二项势能部分与时间有关。当体系有着随时间变化的势场时,哈密顿量就会包含时间。这也就是含时问题的来源。本章我们先讨论哈密顿量不含时的情况;在后面的部分,我们希望探讨哈密顿量含时间时的薛定谔方程的解法。

假若哈密顿量不含时,我们既可以认为是系统本身的哈密顿量恒与时间无关,即定态,也可以认为是在时间演化的观点下研究某一瞬间。这时薛定谔方程写为
\begin{equation}\label{eqn:tdse}
i\hbar \dfrac{\pd \psi(\br,t)}{\pd t}=\bH(\br) \psi(\br,t)
\end{equation}
通常我们将波函数中的时间部分分离来求解薛定谔方程,设
\begin{equation}\label{key}
\psi(\br,t)=u(\br)f(t)
\end{equation}
代入\eqref{eqn:tdse}分离变量得
\begin{gather}
\bH(\br) u(\br)=E u(\br)\label{eqn:TISE}\\
i\hbar \dfrac{\pd f(t)}{\pd t}=Ef(t)\label{eqn:timepart}
\end{gather}
前者即为\textbf{定态薛定谔方程},解之可以得到一组哈密顿量的本征态$ u_n(\br) $以及对应的能量本征值$ E_n $;后者可以直接解出
\begin{equation}\label{key}
f(t)=Ce^{-\frac{iEt}{\hbar}}
\end{equation}
而积分常数$ C $可以交给$ u(\br) $进行归一化。因此定态薛定谔方程的含时通解为
\begin{equation}\label{key}
\psi(\br,t)=\sum_nc_nu_n(\br)e^{-\frac{iE_nt}{\hbar}}
\end{equation}
这里$ u_n(\br) $是正交归一函数组,在Dirac记号下记为$ |n\ra $;$ c_n $与时间无关且$ \sum_n |c_n|^2=1 $。

此通解意味着假如体系处于哈密顿量的某一本征态$ |n\ra $,则概率密度不随时间变化,且能量的测量值始终为确定值;而假如体系处于叠加态$ |\psi(t)\ra=\sum_nc_ne^{-\frac{iE_nt}{\hbar}}|n\ra $,体系在不同哈密顿量本征态上的布居不随时间变化,因为时间部分的模始终为1:
\begin{equation}\label{key}
\la n|\psi(t)\ra=(c_ne^{-\frac{iE_nt}{\hbar}})^*(c_ne^{-\frac{iE_nt}{\hbar}})=|c_n|^2
\end{equation} 
但是在该叠加态的初态上的布居却显然会发生变化:
\begin{equation}\label{key}
\la\psi(0)|\psi(t)\ra=\sum_nc_n^*(c_ne^{-\frac{iE_nt}{\hbar}})=\sum_n|c_n|^2e^{-\frac{iE_nt}{\hbar}}
\end{equation}
具体的变化行为,即求和的结果,一方面依赖于体系本身($ E_n $),另一方面则与初态在本征态上的布居($ |c_n|^2 $)有关系。

在展开各种具体问题之前,我们先进一步了解量子体系时间演化的理论描述。

\section{时间演化算符}
\begin{framed}
本节参考J. J. Sakurai, Modern QM, Rev ed., 第2章。
\end{framed}
我们从另一个角度来描述量子体系时间演化。

记$ t $时刻体系量子态为$ |\alpha,t\ra $\footnote{更一般的记法为$ |\alpha,t_0;t\ra $,若取$ t_0=0 $则简记为$ |\alpha,t\ra $。对下文算符$ \mU(t) $同理。},则初态为
\begin{equation}\label{key}
|\alpha,0\ra\equiv|\alpha\ra
\end{equation}
我们将由时刻0到时刻$ t $量子态发生的变化归结为算符作用的结果
\begin{equation}\label{ppg}
|\alpha,t\ra=\mU(t)|\alpha\ra
\end{equation}
称之为\textbf{时间演化算符},或者\textbf{传播子}\index{ch@传播子}。

我们观察时间演化算符的性质。首先,算符作用前后量子态保持归一,则时间演化算符一定为酉算符,
\begin{equation}\label{key}
\mU^\dagger(t)\mU(t)=1
\end{equation}
其次,由于时间连续演化,算符也必须有加乘性
\begin{equation}\label{key}
\mU(t_2,t_0)=\mU(t_2,t_1)\mU(t_1,t_0)
\end{equation}
第三,时间是连续变量,那么演化时间微元的算符表示为
\begin{equation}\label{key}
\mU(dt)=1-\dfrac{iHdt}{\hbar}
\end{equation}
其中$ H $为哈密顿量\footnote{推导细节参考Goldstein 1980, 407-8)}。由这一性质可以推出
\begin{equation}\label{key}
\mU(t+dt)-\mU(t)=-\dfrac{iH}{\hbar}dt\mU(t)
\end{equation}
写成偏导数形式为
\begin{equation}\label{key}
i\hbar\dfrac{\pd}{\pd t}\mU(t)=H\mU(t)
\end{equation}
此即\textbf{时间演化算符的薛定谔方程}。将方程两边的算符同时作用于量子态,就得到含时薛定谔方程\eqref{eqn:tdsefull}。也就是说,\underline{若能求得时间演化算符$ \mU(t) $,则相当于求解了含时薛定谔方程}。这种方法比上节中直接求解含时薛定谔方程的方法更加方便。

我们将所有时间演化问题根据其哈密顿量分为三种情况:

第一种:若哈密顿量不含时,则时间演化算符为
\begin{equation}\label{key}
\mU(t)=\exp\left(\dfrac{-iHt}{\hbar}\right)
\end{equation}
作用时算符可以写为
\begin{equation}\label{eqn:1.2.9}
\mU(t)=\sum_n\exp\left(\dfrac{-iE_nt}{\hbar}\right)|n\ra\la n|
\end{equation}
其中态$ |n\ra $为哈密顿量的一组完备的本征态。可以验证作用结果与上节所求得的结果完全一致。

第二种:若哈密顿量含时,但是在任意不同时间下的哈密顿量对易,即$ [H(t_1),H(t_2)]\equiv0 $,则时间演化算符为
\begin{equation}\label{key}
\mU(t)=\exp\left[-\dfrac{i}{\hbar}\int_0^t H(\tau)d\tau\right]
\end{equation}
由于哈密顿量含时,因此其本征态可能随时间变化。

第三种:若哈密顿量含时,且不同时间下的哈密顿量不对易,那么时间演化算符写为\textbf{戴森级数}\index{da@戴森级数}
\begin{equation}\label{key}
\mU(t)=1+\sum_{n=1}^\infty\left(\dfrac{-i}{\hbar}\right)^n\int_0^tdt_1\int_0^{t_1}dt_2\cdots\int_0^{t_{n-1}}dt_n H(t_1)H(t_2)\cdots H(t_n)
\end{equation}

本章我们主要介绍哈密顿量不含时的情况。关于后两种哈密顿量含时的情况将于下一章讨论。

\section{实例:静态哈密顿量叠加态的时间演化}
\begin{framed}
本节参考Nitzan 第2章, pp 57--63。	
\end{framed}

我们以最简单的二能级体系来讨论这个问题。

假设态$ |a\ra $和$ |b\ra $是零阶哈密顿量$ H^{(0)} $的正交归一的本征态,能量本征值分别为$ E_a,\ E_b $。现在该体系中加入一个微扰$ V $,不失一般性,令$ V $的对角元为0,而非对角元可以取复数
\begin{equation}\label{key}
V_{ab}=V_{ba}^*=Ve^{-i\eta}
\end{equation}
这里$ V $是非负实数,$ \eta $为实数。则哈密顿量为
\begin{equation}\label{key}
H=\begin{pmatrix}
E_a&Ve^{-i\eta}\\Ve^{i\eta}&E_b
\end{pmatrix}
\end{equation}
设此时哈密顿量的本征态为$ |+\ra $和$ |-\ra $,其本征值可以由哈密顿量对角化得到
\begin{equation}\label{key}
E_\pm=\dfrac{E_a+E_b\pm\sqrt{(E_a-E_b)^2+4V^2}}{2}
\end{equation}
并解出这两个本征态在零阶表象下的表示
\begin{align}
|+\ra=&\cos\theta e^{-i\eta/2}|a\ra+\sin\theta e^{i\eta/2}|b\ra\\
|-\ra=&-\sin\theta e^{-i\eta/2}|a\ra+\cos\theta e^{i\eta/2}|b\ra
\end{align}
其中
\begin{equation}\label{key}
\theta\equiv\arctan\dfrac{E_+-E_a}{V}=\arctan\dfrac{\Delta-\sqrt{\Delta^2+4V^2}}{2V},\quad \Delta=E_a-E_b
\end{equation}
我们可以反解出此时的零阶本征态
\begin{align}
|a\ra=&e^{i\eta/2}(\cos\theta|+\ra-\sin\theta|-\ra)\\
|b\ra=&e^{-i\eta/2}(\sin\theta|+\ra+\cos\theta|-\ra)
\end{align}
对照\eqref{eqn:1.2.9},我们可以写出两个零阶本征态的含时演化
\begin{align}
|a,t\ra=&e^{i\eta/2}(\cos\theta e^{-\frac{iE_+t}{\hbar}}|+\ra-\sin\theta e^{-\frac{iE_-t}{\hbar}}|-\ra)\\
|b,t\ra=&e^{-i\eta/2}(\sin\theta e^{-\frac{iE_+t}{\hbar}}|+\ra+\cos\theta e^{-\frac{iE_-t}{\hbar}}|-\ra)
\end{align}
假设初态为$ |a\ra $,我们希望观察体系向态$ |b\ra $上转移的概率,即
\begin{equation}\label{key}
P_b(t)=|\la b|a,t\ra|^2=|\sin\theta\cos\theta(e^{\frac{-iE_+t}{\hbar}}-e^{\frac{-iE_-t}{\hbar}})|^2
\end{equation}
代入所有临时变量,得到
\begin{equation}\label{key}
P_b(t)=\dfrac{4|V_{ab}|^2}{(E_a-E_b)^2+4|V_{ab}|^2}\sin^2\left(\dfrac{\Omega_R}{2}t\right)
\end{equation}
其中
\begin{equation}\label{key}
\Omega_R=\frac{1}{\hbar}\sqrt{(E_a-E_b)^2+4|V_{ab}|^2}
\end{equation}
称为\textbf{拉比频率}\index{la@拉比频率}。

由此,我们可以使用时间演化算符解决任何静态哈密顿量体系的时间演化问题。下面我们将处理动态哈密顿量体系的时间演化问题。